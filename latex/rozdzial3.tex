\chapter{Analiza rozwiązań do automatycznej diagnostyki choroby Parkinsona}\label{ch:analiza-rozwiazan}


Diagnoza PD jest powszechnie oparta na obserwacjach lekarskich i ocenie objawów klinicznych, w tym charakterystyce różnorodnych objawów ruchowych.
Rosnąca liczba zachorowań i obniżenie wieku osób będących w grupie ryzyka, skutkuje wzrostem zainteresowania dotyczącym narzędzi, które ułatwiłyby
zarówno codzienne funkcjonowanie pacjentów jak i pracę lekarzy.
Tradycyjne metody diagnostyczne mogą być obarczone subiektywizmem ponieważ opierają się na ocenie ruchów, które są czasami subtelne dla
ludzkiego oka i dlatego trudne do sklasyfikowania, co może przyczynić się do błędnej diagnozy.
Ponadto wczesne objawy niemotoryczne PD mogą być łagodne oraz spowodowane wieloma innymi schorzeniami.
Dlatego też rozpoznanie tej choroby na wczesnym etapie stanowi wyzwanie.

Nie da się nie zauważyć, żę sztuczna inteligencja oraz nowoczesne technologie coraz częściej stają się integralną częścią systemu ochrony zdrowia.
Wspierają lekarzy podczas diagnozy oraz wyboru sposobu leczenia pacjenta,a także pozwalają na monitorowanie choroby.
Aby rozwiązać trudności i udoskonalić procedury diagnozowania oraz oceny PD, wdrożono metody uczenia maszynowego do klasyfikacji PD i osób zdrowych lub
pacjentów z podobnymi objawami klinicznymi (np. zaburzeniami ruchu lub innymi zespołami parkinsonowskimi).


Do roku 2020 powstało co najmniej 209 artykułów naukowych dotyczących wykorzystania metod uczenia maszynowego w diagnostyce i monitorowaniu \cite{ML_for_PD_review},
a liczba ta cały czas rośnie.


\section{Wykorzystywane dane}\label{sec:dane-przeglad}

Diagnozowanie choroby Parkinsona stanowi zadanie złożone ze względu na różnorodność objawów, które dotykają różne aspekty
funkcjonowania ciała i umysłu ludzkiego.
W związku z tym, techniki uczenia maszynowego wykorzystywane w tym obszarze również skupiają się na różnych rodzajach danych.
Wśród tych źródeł informacji znajdują się wyniki badań obrazowych (np. rezonans magnetyczny - MRI, tomografia komputerowa - SPECT),
które wydają się intuicyjne, biorąc pod uwagę zmiany w aktywności mózgu, które można zaobserwować.
Niemniej jednak, istnieją także mniej oczywiste metody diagnozy, które budzą duże zainteresowanie w środowisku naukowym, szczególnie w początkowym stadium choroby.
Przykłady to analiza cezury głosu, ocena charakterystyki chodu oraz badanie pisma odręcznego.

\begin{figure}[htbp]
	\centering
	\includegraphics[width=0.5\textwidth]{./img/plot_PD_detection_methods}
	\caption{Wykres przedstawiający rozkład rodzaju danych na których bazowały systemy ML do diagnostyki PD (stan na dzień 14 luty 2020) \cite{ML_for_PD_review} }
    \label{fig:pd_detection_methods}\label{fig:figure}
\end{figure}

Rysunek \ref{fig:pd_detection_methods} ilustruje zastosowanie wymienionych metod zarówno w teorii, jak i praktyce.
Metody oparte na obrazowaniu medycznym wykazują wyraźną przewagę w zastosowaniach klinicznych w porównaniu do kontekstu teoretycznego.
Niemniej jednak, to pozostałe metody budzą znacznie większe zainteresowanie ze strony środowiska naukowego.
Szczególnie w przypadku analizy głosu, gdzie rozbieżność między teorią a praktyką jest szczególnie znacząca.
Przyczyny tego zjawiska zostaną dokładniej rozważone w kolejnych fragmentach.

Niniejsza praca dotyczy diagnostyki na podstawie głosu, dlatego ten temat zostanie rozwinięty.
Założeniem dla takich systemów jest zadanie potencjalnemu pacjentowi zadania wokalnego, może to być:
\begin{itemize}[itemsep=0.1pt]
	\item podtrzymywane samogłoski (ang. sustained vowels)
	\item cechy diadochokinetyczne (DDK), mogące mierzyć zdolność do wydawania serii szybkich i naprzemiennych dźwięków (sylab).
	\item czytanie tekstu
	\item monolog
\end{itemize}

[TUTAJ TRZEBA O TYM NAPISAĆ - WADY, ZALETY ITD]

%---------------------------------------------------------------------------

\section{Metody klasyfikacji}\label{sec:metody-klasyfikacji}

W zależności od przyjętego podłoża diagnostycznego, selekcjonuje się odpowiednią metodę klasyfikacji.
Każdy problem wymaga unikalnego podejścia, a wybór metody jest dostosowany do konkretnej sytuacji.
Różne metody okażą się skuteczne w przypadku analizy mowy spontanicznej w porównaniu do mowy kontrolowanej.
W tej sekcji zostaną przedstawione metody klasyfikacji związane z podtrzymywaniem samogłosek, które to zostały wybrane jako
fundament badawczy w ramach niniejszego opracowania.

%---------------------------------------------------------------------------

\section{Problemy związane z systemami automatycznej klasyfikacji choroby Parkinsona}\label{sec:problemy}

Odnieść się do tego \ref{fig:pd_detection_methods} i duża dysproporcja między teorią i praktyką

Ostatnie badania wykazały, że możemy wytrenować dokładne modele do wykrywania oznak PD z nagrań audio.
Jednakże, istnieją rozbieżności pomiędzy badaniami i mogą być spowodowane, częściowo, przez różnice w
wykorzystywanych korpusach lub metodologii.
Dlatego w  \cite{SustainedVowelsProblems} przeprowadzono analizę, wpływu niektórych czynników na wyniki klasyfikacji.
Głównym celem artykułu była ich identyfikacja oraz stworzenie zasad, które w przyszłosći pozwolą usystematyzować
stan wiedzy w tej dziedzinie.
W badaniach skupiono się na przedłużonych samogłoskach (ang. \emph{sustained vowels}), ponieważ jak wykazano wcześniej
[...].
Przeprowadzone eksperymenty wykazały, że nieuwzględnione zmienne w metodologii, projekcie eksperymentalnym i
przygotowaniu danych prowadzą do zbyt optymistycznych wyników w badaniach nad automatyczną detekcją PD.
Czynniki, które zidentyfikowano jako przyczyniające sią do zbyt optymistycznych wyników klasyfikacji
przedstawiono na Rys. \ref{fig:factors_PD_detection} oraz omówiono poniżej.


\begin{figure}[htbp]
	\centering
	\includegraphics[width=0.4\textwidth]{./img/influence_of_factors_on_PD_detection}
	\caption{Czynniki wpływające na dokładność detekcji Parkinsona na podstawie głosu według analizy przeprowadzonej w \cite{SustainedVowelsProblems}}
    \label{fig:factors_PD_detection}
\end{figure}


\begin{enumerate}[label={\alph*)}]
	\item \textbf{Wpływ tożsamości mówcy}
	\item[] W przypadku, gdy w zbiorze danych znajduje się kilka nagrań od tego samego mówcy można postąpić na dwa sposoby.
Pierwszy z nich to podział według podmiotów (ang. \emph{subject-wise split}) polegający na tym, że nagrania od tej samej
osoby znajdują się albo w zbiorze treningowym albo testowym - nigdy w obu na raz.
W niektórych opracowań można spotkać też drugie podejście, czyli podział według rekordów (ang.\emph{record-wise split}), gdzie nagrania są losowo dzielone do zbiorów
lub intencjonalnie używa się nagrań od tej samej osoby zarówno w zbiorze testowym jak i treningowym.
Okazuje się, że podejście typu \emph{record-wise} prowadzi do wyższej dokładności niż \emph{subject-wise split}, jeśli pozostałe założenia pozostają identyczne.
Prawdopodobnie wynika to z faktu, że klasyfikator nastawia się na wykrywanie unikalnych informacji indywidualnych,
reprezentowanych przez współczynniki takie jak MFCC i PLP, a nie rzeczywiste biomarkery lub wzorce PD.
Dlatego też rekomendowana jest technika \emph{subject-wise split}, aby uniknąć zbyt optymistycznych wyników.

  	\item \textbf{Wpływ różnicy wieku między klasami}
	\item[] W literaturze można znaleźć prace wykorzystujące zbiory danych, w których  średni wiek mówców
w klasie osób chorych na PD różni się od średniego wieku w klasie osób zdrowych o ponad 5 lat.
Autorzy zapewniają o wysokiej skuteczności  swoich rozwiązań, jednak nie rozważają oni sytuacji, w której klasyfikator
uczy się wykrywać cechy powiązane z wiekiem, zamiast rzeczywistych wzorców PD.
Aby zbadać wpływ średniej różnicy wieku między dwiema klasami, w \cite{SustainedVowelsProblems} testowano różne przesunięcia rozkładu wieku.
Wraz ze wzrostem różnicy między średnim wiekiem uczestników z PD i HC, dokładność klasyfikacji konsekwentnie rosła (Rys. \ref{fig:acc_and_age_diff}).
Na tej podstawie można stwierdzić, że związany z wiekiem wpływ na głos mówców może zaburzać wyniki otrzymywane przez klasyfikator.
Dlatego też zaleca się zbilansowanie używanych zbiorów danych, tak aby średnia różnica wieku między dwoma klasami  była jak namniejsza.


\begin{figure}[htbp]
	\centering
	\includegraphics[width=0.7\textwidth]{./img/acc_and_age_difference}
	\caption{Wykres przedstawiający zależność różnicy wieku między klasami a dokładnością klasyfikacji \cite{SustainedVowelsProblems}}
    \label{fig:acc_and_age_diff}
\end{figure}

  	\item Wpływ losowości cech na dokładność klasyfikacji
	\item [] Im większa różnica między liczbą plików a wymiarem wektora cech, tym większe szanse na znalezienie cechy, która losowo koreluje z etykietami klas.
  	\item Łagodzenie losowego nadmiernego dopasowania przy użyciu danych programistycznych
	\item Wpływ rozpoczęcia i przesunięcia samogłosek na wyniki klasyfikacji
 	\item Eksperymenty międzykorporowe
	\item Analiza cech
\end{enumerate}



Nie są to jednak wszystkie czynniki, które zaburzają obiektywność wyników. Konieczna jest dyskusja na temat nowych
kompleksowych linii bazowych dla prowadzenia eksperymentów w automatycznym wykrywaniu PD na podstawie fonacji,
a także innych ogólnych zastosowań przetwarzania mowy.


\subsection{Wnioski}
\label{subsec:wnioski}

Prace nad automatyczną detekcją Parkinsona na podstawie głosu trwają już od dłuższego czasu.
Jednak wciąż brakuje systemu, który mógłby zostać uznany jako wystarczajaco niezawodne narzędzie diagnostyczne.
Wśród problemów, które ograniczają rzeczywiste wykorzystanie takich systemów wyróżnia się:

\begin{itemize}
\item pierwszy punkt
\item drugi punkt
\item trzeci punkt
\end{itemize}


%---------------------------------------------------------------------------
\section{Wykorzystanie rozwiązań teoretycznych w rzeczywistym środowisku}\label{sec:teoria-vs-praktyka}


%---------------------------------------------------------------------------
