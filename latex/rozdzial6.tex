\chapter{Analiza i interpretacja wyników\@}
\label{ch:interpretacja-wynikow}

Przedstawione wyniki, mimo że nie osiągnęły jeszcze poziomu zadowalającego dla zastosowań w rzeczywistych warunkach, stanowią obiecujący punkt wyjścia.
Warto podkreślić, że analiza została przeprowadzona na niewielkiej grupie badawczej, co wprowadza pewne ograniczenia.
Zebrano nagrania od blisko 200 uczestników zróżnicowanych pod względem narodowości, w tym 103 osoby z diagnozą choroby Parkinsona.
Jednakże, aby pełniej oddać różnorodność i reprezentatywność populacji, konieczne jest zgromadzenie większej liczby próbek.

Podjęcie tego wyzwania jest kluczowe, gdyż badanie głębokich metod uczenia maszynowego w kontekście diagnostyki choroby Parkinsona to złożone zagadnienie badawcze.
Analiza literatury naukowej wskazuje, że istnieją możliwości konstrukcji bardziej efektywnych narzędzi, korzystających z zaawansowanych metod uczenia maszynowego.

Mimo to, uzyskany wynik na poziomie 75\% jest bardzo obiecujący i otwiera perspektywy wykorzystania głębokich metod uczenia maszynowego w diagnozowaniu choroby Parkinsona.
Dalsze badania, rozszerzenie bazy danych oraz ulepszenia w modelach mogą przyczynić się do osiągnięcia jeszcze lepszych rezultatów w przyszłości, co byłoby znaczącym osiągnięciem w dziedzinie medycyny.

Warto podkreślić, że badania opierały się na połączeniu trzech zupełnie różnych baz danych, które reprezentowały różne narodowości.
Ten interdyscyplinarny i międzynarodowy charakter podejścia wskazuje na potencjał stworzenia narzędzia diagnostycznego o uniwersalnym zastosowaniu.
To otwiera perspektywy na rozwinięcie narzędzia, które będzie skuteczne w diagnozowaniu choroby Parkinsona niezależnie od narodowości pacjentów.
Ten aspekt pracy jest niezwykle obiecujący i daje nadzieję na przyszłość również z biznesowego punktu widzenia.

Przeprowadzone badania umożliwiają również wstępną analizę przydatności różnych samogłosek i architektur CNN, co zostanie rozwinięte w dalszej części rozdziału.

\section{Samogłoski a informacja diagnostyczna\@}
\label{sec:samogloski-informacja-diagnostyczna}

Przeprowadzone badania wykazały, że najlepsze wyniki osiągnięto dla samogłoski /a/ (75\%), /e/ (73\%) oraz /u/ (72\%), a znacznie słabsze dla /i/ (64\%) oraz /o/ (65\%).
Połączenie wszystkich analizowanych samogłosek w jeden zbiór nie przyniosło poprawy wyników, mimo znaczącego zwiększenia zbioru uczącego.
Może to wiązać się z różnym sposobem przekazywania przez nie informacji diagnostycznej.

W badaniu~\cite{vowels-in-PD} przeanalizowano nagrania samogłosek /a/, /e/, /i/ oraz /u/ u 7 pacjentów i 7 pacjentek.
Analiza wykazała, że charakterystyka tych nagrań różni się zależnie od płci, ale również od wypowiadanej samogłoski.
Biorąc pod uwagę powyższe, rozwiązanie do automatycznej diagnostyki PD powinno skupiać się na jednej lub uwzględniać informację o rodzaju wykorzystanej samogłoski.

Nie spotkano się do tej pory z publikacją, która w bezpośredni i rzetelny sposób porównywałaby potencjał poszczególnych samogłosek w kontekście diagnostyki choroby Parkinsona.
Natomiast w badaniu, którego głównym celem była ocena augmentacji danych~\cite{augmentation} w tym zagadnieniu, pośrednio oceniono potencjał różnych samogłosek w diagnostyce.
Najlepsze wyniki osiągnięto przy wykorzystaniu samogłoski /a/ (92\%), a najgorsze w przypadku /i/ (74\%).
Zależność ta pokrywa się z przeprowadzonymi w tej pracy eksperymentami.
Ponadto samogłoska /a/ jest najczęściej wykorzystywana w podobnych badaniach, więc gdyby istniała potrzeba wskazania najlepszej podstawy diagnostycznej spośród samogłosek to wskazano by samogłoskę /a/, mimo że jej przewaga nad /e/ nie jest znacząca według przeprowadzonych badań.
Oczywiście konieczne jest przeprowadzenie dalszych badań, z wykorzystaniem znacznie większej bazy danych.
Mimo to każda z samogłosek wykazuje potencjał wykorzystania w rozwiązaniu do automatycznej diagnostyki choroby Parkinsona na podstawie głosu.

\section{Architektury CNN a skuteczność klasyfikacji\@}
\label{sec:CNN-a-skutecznosc-klasyfikacji}
W niniejszej pracy porównano potencjał wykorzystania 5 różnych klasyfikatorów pod kątem wykorzystania ich jako narzędzia do diagnostyki choroby Parkinsona na podstawie nagrań głosowych samogłosek.
Ze względu na niewielką ilość danych użyto podejścia znanego jako transfer learning, którego główną zaletą jest korzystanie ze wstępnie wytrenowanych wag, zamiast losowych wartości.
Takie podejście było już badane w literaturze naukowej i wykazało duży potencjał.
Dlatego zdecydowano się wybrać 5 różnych architektur CNN i przeprowadzić ich porównanie.
Ocena różnych architektur sieci neuronowych w kontekście detekcji anomalii w sygnale mowy u osób z chorobą Parkinsona stanowi ważny aspekt badawczy.
Otrzymane wyniki wskazują na istotne różnice w skuteczności tych architektur, co może dostarczyć cennych wskazówek dla przyszłych badań w tej dziedzinie.

Na podstawie przeprowadzonych badań można stwierdzić, że VGG16 wyróżnił się jako najlepsza architektura, osiągając wysokie wyniki dla 4 z 6 testowanych wariantów samogłosek (/a/, /e/, /o/ oraz /u/).
Może to sugerować, że jest ona szczególnie skuteczna w wykrywaniu anomalii w sygnale mowy związanych z chorobą Parkinsona w przypadku samogłosek.
VGG16 to głęboka architektura konwolucyjna, która składa się z 16 warstw konwolucyjnych i warstw poolingowych.
Ta głębokość pozwala na efektywne wykrywanie i ekstrakcję cech złożonych, co może być istotne w rozpoznawaniu subtelnych zmian w sygnale mowy związanych z chorobą Parkinsona.
Ponadto wiele warstw konwolucyjnych jest stosowanych po sobie.
To umożliwia wielokrotne przekształcanie sygnału wejściowego, co pozwala na wyodrębnienie bardziej abstrakcyjnych cech.
W przypadku sygnału mowy może to pomóc w wykryciu subtelnych zmian w akustyce lub innym aspekcie sygnału związanym z chorobą Parkinsona.
VGG16 radzi sobie dobrze w przypadku większości wariantów samogłosek, może to sugerować, że ma zdolność do generalizacji cech istotnych dla detekcji anomalii w sygnale mowy osób z chorobą Parkinsona.
To jest ważne, ponieważ różne warianty samogłosek mogą mieć różne cechy akustyczne.

Xception i MobileNet również wykazały się obiecująco, osiągając dobre wyniki w niektórych przypadkach (odpowiednio /o/ oraz wszystkie samogłoski).
Choć nieco mniej skuteczne niż VGG16, nadal mogą być rozważane jako przyzwoite alternatywy.
Ich dalsza eksploracja może zaowocować jeszcze lepszymi wynikami.

ResNet okazał się mniej skuteczny w porównaniu do pozostałych architektur, co może sugerować, że ta konkretna architektura może nie być odpowiednia do tego konkretnego zadania.
Co ciekawe, obiecujące wyniki wykazał dla samogłoski /a/ i był już  z powodzeniem wykorzystywany w innych badaniach dotyczący tego zagadnienia.

Inception wydaje się mieć wyniki średnie, co sugeruje, że może być skuteczna w niektórych przypadkach, ale niekoniecznie jest najlepszą opcją w każdym scenariuszu.

Należy zwrócić uwagę, że efektywność tych architektur może zależeć od konkretnej charakteryzacji sygnału mowy i konkretnego rodzaju anomalii związanych z chorobą Parkinsona.
Dlatego też warto kontynuować badania w celu dalszej optymalizacji i dostosowania tych architektur do specyfiki tego zadania.

Warto również wziąć pod uwagę, że dalszy \emph{fine-tuning}  tych architektur mógłby jeszcze poprawić skuteczność.
Ponadto eksploracja innych architektur sieci neuronowych lub technik uczenia maszynowego może również przynieść korzyści w dalszych badaniach w tej dziedzinie.

Przeprowadzone badania dostarczają istotnych informacji na temat skuteczności różnych architektur sieci neuronowych w detekcji anomalii w sygnale mowy osób z chorobą Parkinsona.
Wskazują one na potencjalnie obiecujące wyniki VGG16, Xception i MobileNet, ale sugerują również potrzebę dalszych badań i optymalizacji w tej dziedzinie.
Warto jednak zaznaczyć, że wybór architektury zależy od konkretnego zadania i zbioru danych.
Wyniki mogą różnić się w zależności od charakterystyki danych i konkretnej metody przetwarzania sygnału mowy.
Dlatego eksperymentowanie z różnymi architekturami i optymalizacja modelu dla konkretnego zadania jest ważnym krokiem w badaniach nad detekcją anomalii w sygnale mowy u osób z chorobą Parkinsona.

\section{Kierunki rozwoju}
\label{sec:kierunki-rozwoju}

Diagnostyka choroby Parkinsona na podstawie analizy głosu stanowi fascynujące pole badań, które obiecuje rewolucję w wykrywaniu tej choroby w jej wczesnym stadium.
Niniejsza praca nie stanowi rozwiązania, które w zadowalającym stopniu mogłoby służyć jako narzędzie diagnostyczne.
Natomiast bada potencjał wykorzystania metod głębokiego uczenia maszynowego w tej dziedzinie.
Osiągnięte wyniki wskazują bardziej kierunki dalszej eksploracji zagadnienia niż odpowiedź na stawiane pytania.

Pierwszym istotnym kierunkiem rozwoju jest zwiększenie rozmiaru bazy danych.
Przeprowadzone badania bazowały na próbkach od pacjentów z różnych narodowości, jednakże, aby narzędzie to stało się bardziej uniwersalne, konieczne jest uwzględnienie większej liczby danych z różnych kultur i grup wiekowych.
Rozszerzenie bazy danych pozwoli na uzyskanie bardziej reprezentatywnego zbioru, co może istotnie wpłynąć na zdolność modeli do diagnozowania choroby Parkinsona.
Markery choroby Parkinsona są zagadnieniem bardzo skomplikowanym i nie da się na podstawie 100 pacjentów stworzyć uniwersalnego i skutecznego narzędzia diagnostycznego.
Należy również pamiętać, że zmiany w głosie nie tylko dotyczą jedynie 89\% osób chorych, ale też objawiają się w różnym nasileniu na różnych etapach choroby.
Pogłębiona augmentacja danych może przyczynić się do stworzenia bardziej zróżnicowanego zbioru treningowego, co jest kluczowe w kontekście generalizacji modeli.
Rozważenie bardziej zaawansowanych technik augmentacji, takich jak zmienne warunki akustyczne czy emulacja różnych akcentów, może poprawić zdolność modeli do rozpoznawania subtelnych zmian w głosie pacjentów.

Kolejnym krokiem jest eksploracja różnych ustawień spektrogramów i mel-spektrogramów.
Parametry takie jak rozdzielczość czasowa i częstotliwościowa mają znaczący wpływ na jakość danych wejściowych dla modeli.
Dokładna optymalizacja tych ustawień może pomóc w wydobyciu bardziej istotnych informacji z danych i zwiększyć efektywność diagnozy.
Takie eksperymenty zostały wstępnie przeprowadzone na cele tej pracy, jednak konieczne jest ich pogłębienie.
W literaturze również nie ma informacji na temat badań, które definiowałyby tzw.~\emph{złoty środek} tego zagadnienia.
Zarówno jeśli chodzi o wybór między spektrogramami a mel-spektrogramami, ich parametry, jak i sama długość nagrania.

Dla dalszego rozwoju narzędzia diagnostycznego kluczowa jest również eksploracja innych i architektur CNN\@.
Oprócz modeli CNN warto rozważyć wykorzystanie zaawansowanych lub specjalizowanych modeli, a także podejść opartych na uczeniu transferowym, być może trenowanych wstępnie na innych zbiorach danych.
Warte rozważenia byłaby dalsza eksploracja podejścia przedstawionego w~\cite{Wodzinski}.
Polega ono na wstępnym trenowaniu modeli na większej bazie danych dotyczącej nagrań głosowych pacjentów z różnymi schorzeniami (w tym przypadku była to baza SVD), a następnie dotrenowanie modelu na danych dotyczących choroby Parkinsona.
Kombinacja różnych architektur również może pomóc w osiągnięciu wyższej dokładności diagnozy.
Eksperymentowanie z technikami uczenia zespołowego, które łączą wyniki z różnych modeli, może przynieść korzyści w postaci poprawy ogólnej wydajności i niezawodności diagnostyki.
Uczenie zespołowe pozwala na korzystanie z różnych perspektyw i źródeł informacji, co może znacząco zwiększyć skuteczność narzędzia.

Kontynuacja procesu \emph{fine-tuningu} wykorzystanych w pracy modeli może przynieść dalsze doskonalenie i dostosowanie do specyficznych cezur diagnostycznych.
Dalszy rozwój w tym obszarze może prowadzić do bardziej precyzyjnych i efektywnych narzędzi diagnostycznych.
Dużym problemem podczas prób okazał się overfitting, możliwe, że połączenie rozszerzenia bazy danych wraz z dalszym dostrajaniem modeli przyniosłoby znacznie lepsze efekty.

Ostatnim, ale nie mniej istotnym kierunkiem rozwoju, jest uwzględnienie dodatkowych informacji meta.
Dane takie jak płeć, wiek czy język mówiącej osoby mogą stanowić cenne informacje w procesie diagnostycznym.
Są to informacje, które wpływają na przykład na sposób wydobywania dźwięku i jego wysokość.
Dodanie takich cech mogłoby znacząco ułatwić naukę modelu.
Jednakże, aby uniknąć błędów i utrzymania spójności, konieczne jest uzyskanie takich informacji dla wszystkich próbek w zbiorze danych.

Kierunki rozwoju w diagnostyce choroby Parkinsona na podstawie analizy głosu są obiecujące i otwierają perspektywy na stworzenie narzędzia o globalnym zastosowaniu.
Kontynuacja badań w opisanych obszarach może przyczynić się do osiągnięcia jeszcze lepszych wyników i poprawić jakość opieki nad pacjentami cierpiącymi na tę chorobę.
Komputerowe wykrywanie choroby Parkinsona umożliwiłoby przesiewowe badania populacji oraz częste monitorowanie, dostarczając bardziej obiektywny pomiar objawów, co przyniesie korzyści zarówno pacjentom, jak i dostawcom opieki zdrowotnej.
