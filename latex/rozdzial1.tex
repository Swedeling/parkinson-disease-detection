\chapter{Wprowadzenie} \label{ch:wprowadzenie}

Aktualnie prowadzonych jest wiele prac, które umożliwiłyby stworzenie narzędzia do automatycznej diagnostyki Parkinsona na podstawie sygnału głosowego.

Choroba Parkinsona uważana jest za chorobę cywilizacyjną.

%---------------------------------------------------------------------------

\section{Cel pracy}
\label{sec:celPracy}

Celem pracy jest detekcja choroby Parkinsona na podstawie sygnału głosowego z wykorzystaniem metod uczenia maszynowego.
Obejmuje to dokładny przegląd literaturowy ze szczególnym uwzględnieniem aktualnie najlepszych algorytmów
dostępnych w literaturze.
Na tej podstawie oceniona zostanie skuteczność różnych architektur konwolucyjnych sieci
neuronowych (CNN).
Ponadto dokonana zostanie analiza, która z wypowiadanych przec pacjentów samogłosek niesie ze sobą
najwięcej informacji diagnostycznej.

%---------------------------------------------------------------------------

\section{Zakres pracy}
\label{sec:zakresPracy}

\begin{itemize}
\item
detekcja Parkinsona

\item
testowanie różnych architektur

\item ocena różnych samogłosek
\end{itemize}