\chapter{Wprowadzenie} \label{ch:wprowadzenie}

Choroba Parkinsona (ang. \emph {Parkinson Disease, PD}) to zwyrodnieniowe schorzenie mózgu, które wiąże się z objawami ruchowymi, takimi jak spowolnienie ruchowe,
drżenie, sztywność oraz zaburzenia chodu i równowagi.
Ponadto może prowadzić do różnorodnych powikłań niemotorycznych, obejmujących zaburzenia poznawcze, stany psychiczne,
trudności ze snem oraz dolegliwości sensoryczne, w tym ból.
Początkowe objawy często rozwijają się stopniowo, nasilając się w miarę upływu czasu.
Postęp choroby prowadzi do znacznego stopnia niepełnosprawności, co może wymagać wsparcia i opieki.

\begin{figure}[htbp]
	\centering
	\includegraphics[width=0.9\textwidth]{./img/map}
	\caption{Choroba Parkinsona na świecie~\cite {global_PD}}
    \label{fig:PD_map}
\end{figure}

Zgodnie z danymi przedstawionymi w raporcie Światowej Organizacji Zdrowia~\cite{WHO}, choroba Parkinsona stanowi obecnie narastający problem na skalę światową.
Zarówno wskaźniki niepełnosprawności, jak i zgony związane z tą chorobą rosną szybciej niż w przypadku innych zaburzeń neurologicznych.

W ciągu ostatnich 25 lat zaobserwowano podwojenie częstości występowania PD na całym świecie.
Globalne szacunki na rok 2019 wskazują, że liczba osób cierpiących na PD przekroczyła 8,5 miliona, co oznacza wzrost o 81\% w porównaniu z danymi z roku 2000.
Jednocześnie liczba zgonów związanych z PD wyniosła 329~000, co stanowi wzrost o ponad 100\% w porównaniu z rokiem 2000~\cite{global_PD}.
W Polsce z chorobą tą zmaga się około 100 tys.
pacjentów, z czego około 20\% jest już w stadium zaawansowanym
według informacji przekazywanych przez Fundację Chorób Mózgu.
Ponadto co roku w naszym kraju wykrywanych jest ok.
8 tys.
nowych zachorowań.

PD jest istotną sprawą dotyczącą zdrowia publicznego, ponieważ jej częstotliwość występowania związana jest ze zjawiskiem starzejącego się społeczeństwa (Rys.~\ref{fig:PD_map}).
Razem z innymi chorobami neurodegeneracyjnymi, takimi jak choroba Alzheimera, ma szanse stać się drugą zaraz za nowotworami przyczyną zgonów do 2040 roku (WHO).

\begin{figure}[htbp]
	\centering
	\includegraphics[width=0.5\textwidth]{./img/PD_prevalence}
	\caption{Rozpowszechnienie choroby Parkinsona w zależności od wieku \cite {global_PD}}
    \label{fig:PD_prevalance}
\end{figure}

Każdy może być narażony na ryzyko rozwoju choroby Parkinsona, ale częściej występuje ona u mężczyzn niż u kobiet (Rys.\ref{fig:PD_prevalance}).
Statystyki pokazują, że ryzyko zachorowania rośnie wraz z wiekiem, chociaż może ona dotyczyć także młodszych osób (nawet w wieku 20 lat).
U większości pacjentów po raz pierwszy choroba rozwija się po 60 roku życia, około 5\% do 10\% doświadcza jej początku przed 50 rokiem życia.

Przyczyna PD nie jest znana, ale uważa się, że powstaje w wyniku złożonej interakcji pomiędzy czynnikami genetycznymi i
narażeniem na czynniki środowiskowe, takie jak pestycydy, rozpuszczalniki i zanieczyszczenia powietrza.
Niektóre przypadki wydają się dziedziczne, a kilka można przypisać określonym wariantom genetycznym.
Chociaż uważa się, że genetyka odgrywa rolę w chorobie Parkinsona, to w większości przypadków nie występuje ona  rodzinnie~\cite{National_Institute_on_Aging_2022}.

Proces diagnozowania choroby jest niezwykle złożony i czasochłonny.
Nie istnieje obecnie kompletne badanie pozwalające na postawienie pewnej diagnozy.
W związku z tym poszukuje się nowych rozwiązań, które mogłyby usprawnić ten proces.
Coraz częściej wykorzystuje się metody uczenia maszynowego i sztucznej inteligencji w dziedzinie medycyny.
W ramach niniejszej pracy przeanalizowano aktualny stan rzeczy oraz zbadano potencjał jednego z proponowanych w literaturze zrozwiązań, dotyczącego automatycznej diagnostyki
choroby Parkinsona na podstawie głosu.

%---------------------------------------------------------------------------

\section{Cel pracy}
\label{sec:celPracy}

Celem pracy jest detekcja choroby Parkinsona na podstawie sygnału głosowego z wykorzystaniem metod uczenia maszynowego.
Obejmuje to dokładny przegląd literaturowy ze szczególnym uwzględnieniem aktualnie najlepszych algorytmów
dostępnych w literaturze.
Pośrednim celem jest ocena skuteczności wybranych architektur konwolucyjnych sieci neuronowych (CNN) oraz analiza, która z wypowiadanych przez
pacjentów samogłosek niesie ze sobą najwięcej informacji diagnostycznych.

%---------------------------------------------------------------------------

\section{Zakres pracy}
\label{sec:zakresPracy}
