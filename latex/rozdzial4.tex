\chapter{Materiał i metoda badawcza}

W ramach pracy przeprowadzone zostały badania dotyczące klasyfikacji choroby Parkinsona.
Głównym celem było opracowanie efektywnego modelu klasyfikacji binarnej, który może rozróżniać
osoby chore na chorobę Parkinsona od osób zdrowych na podstawie analizy sygnałów mowy.
Przyjęte podejście opiera się na wykorzystaniu metod przetwarzania sygnałów mowy oraz
uczenia maszynowego. Najpierw zebrano dane, w tym nagrania głosowe osób z chorobą Parkinsona
(PD) oraz osób zdrowych (HC, ang. \emph{Healthy Controls}). Następnie przeprowadzono analizę przy użyciu różnych ustawień
spektrogramów i melspektrogramów. Obejmuje to zmienne parametry takie jak rozmiar okna i
przesunięcie okna. Dodatkowo, przeprowadzono badania dotyczące różnych architektur
konwolucyjnych sieci neuronowych. Celem jest zbadanie, które architektury sieci i ustawienia
spektrogramów dają najlepsze wyniki w klasyfikacji choroby Parkinsona dla poszczególnych
samogłosek. Przeprowadzona analiza porównawcza pozwala na lepsze zrozumienie wpływu tych
czynników na skuteczność klasyfikacji. W rezultacie, możliwe będzie ustalenie optymalnych ustawień i
architektur dla klasyfikacji choroby Parkinsona na podstawie analizy sygnałów mowy. Ponadto zbadano
wpływ rozszerzenia zbioru danych o dodatkowe nagrania pochodzące od tych samych osób.

\begin{figure}[htbp]
	\centering
	\includegraphics[width=1\textwidth]{./img/methodology}
	\caption{Schemat przyjętej metody badawczej [opracowanie własne]}
    \label{fig:methodology}
\end{figure}

%---------------------------------------------------------------------------

\section{Materiał badawczy}
\label{sec:material-badawczy}

Materiałem badawczym w niniejszej pracy magisterskiej są nagrania głosowe samogłosek: /a/,
/e/, /i/, /o/ oraz /u/. W zbiorze danych znajdują się nagrania osób zdrowych oraz z chorobą
Parkinsona.

[Opis źródła nagrań osób chorych]

Nagrania osób zdrowych zostały zebrane w ramach pracy przy wykorzystaniu aplikacji Easy
Voice Recorder, która jest programem do nagrywania dźwięku. Przed przystąpieniem do pozyskiwania
danych zapoznano się z charakterystyką problemu. Na podstawie literatury wyróżniono płeć, wiek oraz
język ojczysty jako czynniki, które mogą wpływać na wyniki klasyfikacji. Zbiór danych powinien być
zrównoważony pod kątem tych aspektów, tak by nie dopuścić do sytuacji, gdy model uczy się wzorców
nie związanych z chorobą Parkinsona.
Ustalono protokół nagrywania, wykluczając osoby poniżej 50 roku życia, palące oraz ze
zdiagnozowaną lub podejrzewaną chorobą, która wpływa na aparat mowy lub korę mózgową (np.
choroba Parkinsona, epilepsja, padaczka). Pozyskiwano nagrania jedynie od osób, dla których językiem
ojczystym jest polski. W tabeli \ref{tab:charakterystyka-bazy-danych} przedstawiono informacje dotyczące bazy danych.


\begin{table}[t]
\centering
\caption{Charakterystyka bazy danych [opracowanie własne]}
\label{tab:charakterystyka-bazy-danych}
\begin{tabular}{|l|c|c|c|}
\hline
\textbf{Kategoria} &\textbf{Osoby zdrowe (HC)} &\textbf{Osoby chore (PD)} &\textbf{Razem} \\ \hline
Liczba osób &26 &27 &53\\ \hline
Średnia wieku &60,88 ± 7,98 &64,49 ± 8,49  &62,68 ± 8,43\\ \hline
Liczba kobiet &18 &13 &31\\ \hline
Liczba mężczyzn &8 &14 &22 \\ \hline
\end{tabular}
\end{table}

%---------------------------------------------------------------------------

\section{Parametryzacja sygnału akustycznego}
\label{sec:parametryzacja-sygnalu-akustycznego}

%---------------------------------------------------------------------------

\section{Metody klasyfikacji}
\label{sec:klasyfikacja}

%---------------------------------------------------------------------------

\section{Metody ewaluacji wyników}
\label{sec:metody-ewaluacji-wyników}