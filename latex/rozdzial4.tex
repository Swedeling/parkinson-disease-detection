\chapter{Materiał i metoda badawcza}

W ramach pracy przeprowadzone zostały badania dotyczące klasyfikacji choroby Parkinsona.
Głównym celem było opracowanie efektywnego modelu klasyfikacji binarnej, który może rozróżniać
osoby chore na chorobę Parkinsona od osób zdrowych na podstawie analizy sygnałów mowy.
Przyjęte podejście opiera się na wykorzystaniu metod przetwarzania sygnałów mowy oraz
uczenia maszynowego. Najpierw zebrano dane, w tym nagrania głosowe osób z chorobą Parkinsona
(PD) oraz osób zdrowych (HC, ang. \emph{Healthy Controls}). Następnie przeprowadzono analizę przy użyciu różnych ustawień
spektrogramów i melspektrogramów. Obejmuje to zmienne parametry takie jak rozmiar okna i
przesunięcie okna. Dodatkowo, przeprowadzono badania dotyczące różnych architektur
konwolucyjnych sieci neuronowych. Celem jest zbadanie, które architektury sieci i ustawienia
spektrogramów dają najlepsze wyniki w klasyfikacji choroby Parkinsona dla poszczególnych
samogłosek. Przeprowadzona analiza porównawcza pozwala na lepsze zrozumienie wpływu tych
czynników na skuteczność klasyfikacji. W rezultacie, możliwe będzie ustalenie optymalnych ustawień i
architektur dla klasyfikacji choroby Parkinsona na podstawie analizy sygnałów mowy. Ponadto zbadano
wpływ rozszerzenia zbioru danych o dodatkowe nagrania pochodzące od tych samych osób.

\begin{figure}[htbp]
	\centering
	\includegraphics[width=1\textwidth]{./img/methodology}
	\caption{Schemat przyjętej metody badawczej [opracowanie własne]}
    \label{fig:methodology}
\end{figure}

%---------------------------------------------------------------------------

\section{Materiał badawczy}
\label{sec:material-badawczy}

Materiałem badawczym w niniejszej pracy magisterskiej są nagrania głosowe samogłosek: /a/,
/e/, /i/, /o/ oraz /u/.
Baza danych obejmuje nagrania osób zdrowych oraz z chorobą  Parkinsona.

[Opis źródła nagrań osób chorych]
Autorka pracy wykonała nagrania głosu u osób z chorob ˛a Parkinson’a w Krakowskim Szpitalu Specjalistycznym im. Jana Pawła II. Nagrania obejmowały wypowiedzenie samogłosek /a/,
/e/, /i/, /o/ oraz /u/ w przedłuzonej intonacji przez 27 pacjentów. Pierwszy pomiar wyst˛epował ˙
wtedy, kiedy u pacjenta zdiagnozowano zupełne wysycenie leków łagodz ˛acych objawy choroby. Nast˛epnie lekarz podawał lek lewodopy. Kolejne pomiary wykonano po 30, 60, 120 i 180
minutach od spozycia leku. Przed ka ˙ zdym pomiarem, lekarz neurolog wykonywał badanie i ˙
opisywał stan pacjenta w skali UPDRS.

Nagrania osób zdrowych zostały zebrane w ramach pracy przy wykorzystaniu aplikacji Easy
Voice Recorder, która jest programem do nagrywania dźwięku. Przed przystąpieniem do pozyskiwania
danych zapoznano się z charakterystyką problemu. Na podstawie literatury wyróżniono płeć, wiek oraz
język ojczysty jako czynniki, które mogą wpływać na wyniki klasyfikacji. Zbiór danych powinien być
zrównoważony pod kątem tych aspektów, tak by nie dopuścić do sytuacji, gdy model uczy się wzorców
nie związanych z chorobą Parkinsona.
Ustalono protokół nagrywania, wykluczając osoby poniżej 50 roku życia, palące oraz ze
zdiagnozowaną lub podejrzewaną chorobą wpływającą na aparat mowy lub korę mózgową (np.
choroba Parkinsona, epilepsja, padaczka).
Pozyskiwano nagrania jedynie od osób, dla których językiem ojczystym jest polski.
W tabeli \ref{tab:charakterystyka-bazy-danych} przedstawiono informacje dotyczące bazy danych.

\begin{table}[t]
\centering
\caption{Charakterystyka bazy danych [opracowanie własne]}
\label{tab:charakterystyka-bazy-danych}
\begin{tabular}{|l|c|c|c|}
\hline
\textbf{Kategoria} &\textbf{Osoby zdrowe (HC)} &\textbf{Osoby chore (PD)} &\textbf{Razem} \\ \hline
Liczba osób &26 &27 &53\\ \hline
Średnia wieku &60,88 ± 7,98 &64,49 ± 8,49  &62,68 ± 8,43\\ \hline
Liczba kobiet &18 &13 &31\\ \hline
Liczba mężczyzn &8 &14 &22 \\ \hline
\end{tabular}
\end{table}

\begin{figure}[htbp]
	\centering
	\includegraphics[width=1\textwidth]{./img/database}
	\caption{Charakterystyka bazy danych [opracowanie własne]}
    \label{fig:database}
\end{figure}

Starano się zachować jak najbardziej zbliżone proporcje wieku i płci pomiędzy grupą
pacjentów a grupą porównawczą.
Jednak ze względu na specyfikę samodzielnego zbierania danych, nie udało się osiągnąć pełnej zgodności w tym zakresie.
Mimo to zapewniono różnorodny zbiór danych, obejmujący zarówno kobiety, jak i mężczyzn w różnych grupach wiekowych.
Drobne różnice w liczbie próbek w poszczególnych przedziałach wiekowych nie powinny mieć istotnego wpływu na wyniki
klasyfikacji.
Szczegółowa charakterystyka zbioru danych została przedstawiona na Rys. \ref{fig:database}

Każda osoba kwalifikująca się do badania otrzymała zadanie trzykrotnego wypowiedzenia
samogłosek /a/, /e/, /i/, /o/ oraz /u/ w odstępach czasowych, utrzymując dźwięk przez co najmniej 2 sekundy.
Aby zapewnić jednakowe warunki nagrywania, wyeliminowano hałasy pochodzące z
otoczenia oraz wykorzystano pomieszczenia o podobnej akustyce.
Wszystkie nagrania zostały zarejestrowane z częstotliwością próbkowania 44 kHz.
Czas trwania nagrań samogłosek wynosił od 2 do 5 sekund.
Samogłoski były nagrywane trzykrotnie dla każdej z osób.

Nagrania zostały dokładnie przeanalizowane.
Usunięto nagrania zbyt krótkie oraz te, które nie spełniały kryteriów dotyczących jakości.
Otrzymana baza danych nadal zawierała nagrania od 27 osób chorych oraz od 26 osób zdrowych, zmieniła się jedynie liczebność nagrań dla poszczególnych
samogłosek.

Jednym z głównych celów niniejszej pracy jest przeprowadzenie analizy porównawczej
samogłosek pod kątem ich przydatności w klasyfikacji choroby Parkinsona.
Aby zagwarantować wiarygodność wyników, niezwykle istotne jest utrzymanie jak najbardziej zbliżonych warunków
eksperymentalnych.
Kluczowym aspektem tego zagadnienia jest odpowiednie dostosowanie zbiorów
danych, ponieważ może mieć znaczący wpływ na ostateczne rezultaty analizy.
Początkowo dysponowano zbiorem, w którym na każdą osobę zdrową przypadały trzy nagrania każdej z samogłosek, a na osobę cierpiącą na chorobę Parkinsona jedno.
Jednak po oczyszczeniu bazy danych, liczby te uległy zmianie.
W związku z tym, podjęto decyzję o ograniczeniu zbioru w taki sposób, aby dla każdej samogłoski uwzględniona była
identyczna liczba nagrań pochodzących od poszczególnych osób. Ostateczny skład wybranej bazy
danych przedstawiono w tabeli 4.1.2 oraz na wykresie 4.1.3.

%---------------------------------------------------------------------------

\section{Parametryzacja sygnału akustycznego}
\label{sec:parametryzacja-sygnalu-akustycznego}

%---------------------------------------------------------------------------

\section{Metody klasyfikacji}
\label{sec:klasyfikacja}

%---------------------------------------------------------------------------

\section{Metody ewaluacji wyników}
\label{sec:metody-ewaluacji-wyników}