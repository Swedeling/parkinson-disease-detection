\chapter{Podsumowanie}

Choroba Parkinsona to złożone schorzenie mózgu, które wywołuje różnorodne objawy, w tym zaburzenia mowy, aż u 60-80\% pacjentów.
Te deficyty mowy wynikają głównie z problemów z funkcjonowaniem krtani, zmniejszonej pojemności płuc, osłabionych mięśni mimicznych oraz utraty napędu mówienia. Proces diagnozowania tej choroby jest trudny i może trwać średnio aż 2,7 roku, przy czym często obarczony jest subiektywizmem.

Dlatego poszukiwane są nowe metody, które mogą usprawnić proces diagnozowania choroby Parkinsona i poprawić jakość życia pacjentów.
Jednym z obiecujących kierunków są technologie uczenia maszynowego, w tym te oparte na analizie głosu.
W literaturze naukowej dominują badania nad wykorzystaniem głosu, które osiągają konkurencyjne wyniki.
Przedstawiane są różnorodne podejścia, od tradycyjnej inżynierii cech po wykorzystanie konwolucyjnych sieci neuronowych (CNN).
Skuteczność tych metod waha się od 70\% do 100\%, zależnie od zbioru danych i przyjętej metodyki.

Należy jednak pamiętać, że istnieją czynniki, które mogą wpłynąć na wyniki i sprawić, że są one zbyt optymistyczne.
Dotyczy to m.in.
niewłaściwego podziału zbioru danych, problemu niezrównoważenia klas, nadmiernego rozmiaru cech w stosunku do bazy danych oraz braku badań międzykorpusowych.
Zidentyfikowanie i uwzględnienie tych czynników jest kluczowe, aby dokładnie porównać różne rozwiązania.

W ramach pracy utworzono bazę danych z nagraniami w trzech językach: włoskim, hiszpańskim i polskim (łącznie 484 próbki).
Skupiono się na nagraniach samogłosek /a/, /e/, /i/, /o/ oraz /u/ ze względu na ich uniwersalność.
Nagrania zostały poddane oczyszczeniu, przycięciu do 0,1 sekundy i przekształceniu w mel-spektrogramy.
Zastosowano podejście znane jako \emph{transfer learning} i przetestowano je dla 5 różnych architektur CNN: VGG16, ResNet50, InceptionV3, Xception oraz MobileNetV2.
Następnie dostrojone je, aby jak najlepiej dopasować się do danych głosowych.
Stworzone modele oceniono przy użyciu metryk, takich jak dokładność, F1, specyficzność i precyzja.

Wyniki są obiecujące, szczególnie dla samogłosek /a/, /e/ i /u/, gdzie osiągnięto dokładność przekraczającą 70\%.
W przypadku /i/, /o/ oraz połączenia wszystkich samogłosek, wyniki są nieco niższe i utrzymują się na poziomie około 65\%.
Pozostałe metryki są zbliżone, co sugeruje podobną jakość rozpoznawania każdej z klas.
To oznacza, że prawdopodobieństwo błędu I i II rodzaju jest podobne.
Jeśli miałoby to wpływ na wybór najlepszej podstawy diagnostycznej spośród samogłosek, to samogłoska /a/ wydaje się być najbardziej obiecująca (dokładność na poziomie 75\%).

Pod względem architektur, model VGG16 osiągnął najlepsze wyniki w 4 z 6 analizowanych przypadków.
Wyniki Xception i MobileNetV2 były również obiecujące w przypadku samogłosek /i/ i /o/.
Badania dostarczają ważnych informacji na temat skuteczności różnych architektur CNN w detekcji anomalii w sygnale mowy osób z chorobą Parkinsona.
Wskazują one na potencjalnie obiecujące wyniki VGG16, Xception i MobileNet, ale jednocześnie sugerują potrzebę dalszych badań i optymalizacji.

Osiągnięte wyniki  różnią się od najlepszych rozwiązań przedstawionych w literaturze, które często osiągają ponad 90\% dokładności.
Jednak różnice te można tłumaczyć różnicami w przygotowaniu danych.
Ważne jest, aby zwrócić uwagę na aspekt preprocessingu, który często jest pomijany lub niewystarczająco opisywany w literaturze, a który może mieć kluczowe znaczenie dla osiągnięcia wysokiej dokładności.

Dalsze prace nad tym tematem mogą obejmować zwiększenie rozmiaru bazy danych, eksplorację różnych ustawień spektrogramów i mel-spektrogramów oraz kontynuację procesu \emph{fine-tuningu} modeli.
Ponadto warto rozważyć uwzględnienie dodatkowych informacji meta, takich jak płeć, wiek czy język mówiącej osoby, co może usprawnić proces diagnozy.

Badania nad wykorzystaniem analizy głosu w diagnozowaniu choroby Parkinsona są obiecujące i otwierają nowe perspektywy w opiece nad pacjentami cierpiącymi na tę chorobę.
Komputerowe narzędzia do wykrywania choroby mogą przyczynić się do wcześniejszego rozpoznania i lepszego zarządzania nią, co przyniesie korzyści zarówno pacjentom, jak i dostawcom opieki zdrowotnej.