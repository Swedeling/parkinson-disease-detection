\chapter{Podsumowanie}

Osiągnięte wyniki znacznie odbiegają od najlepszych rozwiązań przedstawionych w literaturze wykorzystujących podobne metody (ponad 90\%).
Powodem tej rozbieżności jest prawdopodobnie sposób przygotowania oraz ilość i jakość danych.
Większość publikacji pomija aspekt preprocessingu lub opisuje go w niewystarczającym stopniu, dlatego odtworzenie raportowanych skuteczności nie jest możliwe.
Należałoby poświęcić temu zagadnieniu jeszcze więcej uwagi, ponieważ wydaje się to być to niezwykle ważnym aspektem tego zagadnienia.

